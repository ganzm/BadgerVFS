
\section{The File Format}


This section describes the binary file format used by the file system inside a
virtual disk.
The file is separated into three major parts. The header, index and the data
section. Each of them is described below.

\subsection{Header Section}

\begin{tabular}{|l|l|p{5cm}|}
\hline
  \textbf{Name} & \textbf{Length} & \textbf{Description}
\\  \hline
  Info & 50 byte UTF-8 String & Contains something like Badger VFS 2013 V1.0 
\\ \hline
  Version & 10 byte UTF-8 String & Contains something like "1.0"
\\ \hline
  MaximumSize & long (8 byte) & Size in bytes this file is allowed to grow (-1) if there is no restriction. 
\\ \hline
  Compression used & 20 byte UTF-8 String & null or indicates compression used for this file
\\ \hline
  Encryption used & 20 byte UTF-8 String & null or indicates encryption used for this file
\\ \hline
 DirectorySectionOffset & long (8 byte) &  File offset where our directory section
 starts \\ \hline
 DataSectionOffset & long (8 byte) &  File offset where our data section starts
\\ \hline
 SaltString & 8 bytes  & Salt used to hash username and password randomly string generated while creating this
   file
 \\ \hline
  Password & xxx bytes  & CryptoHash (SHA-whatever) of Password+SaltString
\\ \hline

\end{tabular}


\subsection{Directory Section}

The directory section describes which files and folders belong to which parent directory. This section has a fixed size and contains so called DirectoryBlocks which also have a fixed size. This makes management an manipulation easy. To each directory belongs a B-Tree structure which lists all contained entries.


\subsubsection*{Directory Block}

One Directory Block represents a Node in our B-Tree of order 2.

\begin{tabular}{|l|l|p{5cm}|}
\hline
  \textbf{Name} & \textbf{Length} & \textbf{Description}
\\  \hline

DirectoryHeader & 1 byte & Header information. This header makes it easy to determine whether a DirecotryBlock is in used or not (memory management)

\\  \hline

DirectoryEntryBlock1 & 128 byte & The smaller key inserted into our B-Tree

\\  \hline

DirectoryEntryBlock2 & 128 byte & The bigger key inserted into our B-Tree

\\  \hline

DirectoryBlockLink1 & 8 byte & Points to another DirectoryBlock which contains keys smaller than DirectoryEntryBlock1

\\  \hline

DirectoryBlockLink2 & 8 byte & Points to another DirectoryBlock which contains keys bigger than DirectoryEntryBlock1 but smaller than DirectoryEntryBlock2

\\  \hline

DirectoryBlockLink3 & 8 byte & Points to another DirectoryBlock which contains keys bigger than DirectoryEntryBlock2

\\  \hline


\end{tabular}

\subsubsection*{Directory Entry Block}

Represents a single directory or file.

\begin{tabular}{|l|l|p{5cm}|}
\hline
  \textbf{Name} & \textbf{Length} & \textbf{Description}
\\  \hline

Filename & 112 byte & UTF-8 Filename String 


\\  \hline

DataBlockLocation & 8 byte & Pointer to a DataBlock located in the Data Section.
This DataBlock holds some meta information about the current directory


\\  \hline

DirectoryEntryTreeRoot & 8 bytes & Pointer to a DirectoryBlock located in the
Directory Section. This referenced DirectoryBlock is the Root Block of a B-Tree
containing all entries of that directory specified by the current Directory Entry Block.
\newline

\textbf{This field containing a 0 indicates that this entry is a file not a directory}



\\  \hline

\end{tabular}


\subsection{Data Section}
The data section is split into blocks where each of them is X bytes long.
Each block contains some amount of data and points to a subsequent block


Block layout \\

\begin{tabular}{|l|l|p{5cm}|}
\hline
  \textbf{Name} & \textbf{Length} & \textbf{Description}
\\  \hline
 BlockHeader & 1 byte & 
 \\
 \hspace{0.2cm} 0) Header-Bit (LSB) & &  If set to 1 this is the first datablock
 of a file.
 \\ 
 \hspace{0.2cm} 1) not used & &  
 \\ 
 \hspace{0.2cm} 2) not used & &  
 \\ 
 \hspace{0.2cm} 3) not used & &  
 \\ 
 \hspace{0.2cm} 4) not used & &  
 \\ 
 \hspace{0.2cm} 5) not used & &  
 \\ 
 \hspace{0.2cm} 6) not used & &  
 \\ 
 \hspace{0.2cm} 7) not used & &  
 
\\  \hline
 NextDataBlock & 8 byte & 
 Points to the start address of the next Datablock (linked list).
    0 if this is the last DataBlock of a certain file or folder.
\\  \hline
  CreationDate & 8 byte & UTC Time when this file was created
  \newline \textit{This field only exists if Header-Bit is set to 1}
\\  \hline

  DataLength & 4 byte &
    Indicates the number of data saved on this DataBlock.
    
\\  \hline
 Data & n byte & user data (may be encrypted/compressed)
\\  \hline
\end{tabular}


\subsection{The root directory}


\section{The Compression}

To reduce the data volume within the virtual disk, compression on each file can
be enabled. Currently available compression algorithms are run length encoding
\cite{rle} and LZ77 \cite{lz77}.

\subsection{Run Length Encoding}

The available 8bit run length encoding(rle) algorithm is a very simple form of
data compression where multiple occurrence of the same byte were stored as a
single byte value and the corresponding count. It is useful for simple graphic
images like line drawings and icons.

\subsection{LZ77}

Abraham Lempel and Jacob Ziv introduced the LZ77 lossless compression algorithm
in 1977. Newer compression methods such as GZIP or DEFLATE often use LZ77-based
algorithms. The compression is achieved by replacing the data with a reference
to an earlier existing copy in the data input stream. For that a window of
a certain size is held in memory where existing copies of the current data are
searched.
