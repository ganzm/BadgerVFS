
\section{The File Format}


This section describes the binary file format used by the file system inside a
virtual disk.
The file is separated into three major parts. The header, index and the data
section. Each of them is described below.

\subsection{Header Section}

\begin{tabular}{|l|l|p{5cm}|}
\hline
  \textbf{Name} & \textbf{Lenght} & \textbf{Description}
\\  \hline
  Info & 50 byte UTF-8 String & Contains something like Badger VFS 2013 V1.0 
\\ \hline
  Version & 10 byte UTF-8 String & Contains something like "1.0"
\\ \hline
  Compression used & 20 byte UTF-8 String & null or indicates compression used for this file
\\ \hline
  Encryption used & 20 byte UTF-8 String & null or indicates encryption used for this file
\\ \hline
 IndexSectionOffset & long (8 byte) &  File offset where our index section
 starts \\ \hline
 DataSectionOffset & long (8 byte) &  File offset where our data section starts
\\ \hline
 SaltString & 8 bytes  & Salt used to hash username and password randomly string generated while creating this
   file
 \\ \hline
  Password & xxx bytes  & CryptoHash (SHA-whatever) of Password+SaltString
\\ \hline

\end{tabular}


\subsection{Index Section}

Data in the index section are organized in a B-Tree structure.


\subsection{Data Section}
The data section is split into blocks where each of them is X bytes long. Each block contains some amount of data and points to a subsequent block


Block layout \\

\begin{tabular}{|l|l|p{5cm}|}
\hline
  \textbf{Name} & \textbf{Length} & \textbf{Description}
\\  \hline
 BlockHeader & 1 byte &
 \\
 \hspace{0.2cm} 0) Header-Bit (LSB) & &  If set to 1 this is the first datablock
 of a file.
 \\ 
 \hspace{0.2cm} 1) Directory-Bit & &  If set to 1 this is a directory, not a file
 \\ 
 \hspace{0.2cm} 2) not used & &  
 \\ 
 \hspace{0.2cm} 3) not used & &  
 \\ 
 \hspace{0.2cm} 4) not used & &  
 \\ 
 \hspace{0.2cm} 5) not used & &  
 \\ 
 \hspace{0.2cm} 6) not used & &  
 \\ 
 \hspace{0.2cm} 7) not used & &  
 
\\  \hline
 NextDataBlock & 8 byte long & 
 Points to the start address of the next Datablock (linked list).
    0 if this is the last Data block of a certain file or folder.
\\  \hline
  HeaderLengthIndicator & 4 byte &     indicates the lenght of the DataBlock Header in bytes
  \newline This field only exists if BlockHeader Bit is set to 1
\\  \hline
  Header & 
  n byte &
  Header Informationen creation date, modification date, file name. (May be encrypted/compressed)
  \newline This field only exists if BlockHeader Bit is set to 1
\\  \hline
  DataLenghtIndicator & 4 byte &
    Indicates the number of data saved on this DataBlock
\\  \hline
 Data & n byte & user data (may be encrypted/compressed)
\\  \hline
\end{tabular}