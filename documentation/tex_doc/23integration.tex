\subsection{Integration}
To achieve goals like file synchronization, conflict management and history
browsing of a file, the interfaces developed in the second part of the project
were no longer good enough. Thus the journaling classes were integrated into the
core library. These classes help keeping track of the actions that take place on the
file system and are used to send changes on a local disk to the server. To have
journaling running as efficiently as possible the shallow copy mechanism was
developed. This mechanism does not copy the full file content but just has
reference counting on the datablock. The counters get increased/decreased
as copies are taken into the journal or deleted. This shallow-copy mechanism
only works, because files cannot be changed on the file system, they have to be renamed or deleted before
new content can be stored under the same name.
