\section{Glossary}

\paragraph{VFS core} The main Java library, that handles all the interaction
with virtual disks and importing/exporting/storing files. It is used by the
command line client and the gui.

\paragraph{Virtual Disk} A virtual disk denotes a container file that is stored
on the host file system. A virtual disk can be opened with the software that is
developed during this project and stores the actual files. The file extension of
the virtual disk is ``*.bfs''.


\section{Command line client}

The command line client allows the usage of the VFS core and is mainly intended
to test the basic functionalities. The console runs either in  management mode
or in filesystem mode. The management mode is entered automatically when
starting the command line client. It allows creating and disposing virtual
disks. The filesystem mode is entered as soon as a virtual disk is opened.

\textbf{TODO: DISCUSSION: sollen ganze ordner importiert und exportiert werden
können? wird dies von der client-seite gehandelt?}

\subsection{startup}
The command line client can be started as follows:

\code{ java -jar VFSCore.jar ch.eth.jcd.badgers.vfs.ui.VFSConsole }

\subsection{commands}
Following commands can be used with the command line client in management mode:

\begin{itemize}
  \item{\textbf{create c:\textbackslash path\textbackslash to\textbackslash
  disk.bfs 1024}} creates virtual disk with a maximum quota of 1024 megabytes on the host system. The file may grow up to 1024 megabytes.
  \textbf{TODO}: more parameters are needed (encryption, compression, password if there is encryption)
  \item {\textbf{dispose c:\textbackslash path\textbackslash to\textbackslash
  disk.bfs}} deletes the given virtual disk
  \item {\textbf{open c:\textbackslash path\textbackslash to\textbackslash
  disk.bfs}} opens filesystem mode for the given virtual disk
\end{itemize}

follwing commands can be used in filesystem mode:

\begin{itemize}
  \item {\textbf{ls}} lists the contents of the current directory
  \item {\textbf{rm file}} deletes the entry denoted as file
  \item {\textbf{cp src dst}} copies the src file to dst 
  \item {\textbf{import ext\_dst}} imports a ext\_src from the host system to dst
  \item {\textbf{export src ext\_src}} exports a src file to the host system
  ext\_dst
\end{itemize}

