
% PART IV: Quick Start Guide
% --------------------------------------

\section{Quick Start Guide}


\subsection{the eclipse project}
The project requires to be compiled with JAVA 7. It also depends on the maven
plugin which pulls in all the required libraries.

\subsection{Command line client}

The command line client allows the usage of the VFS core and is mainly intended
to test the basic functionalities. The console runs either in  management mode
or in filesystem mode. The management mode is entered automatically when
starting the command line client. It allows creating and opening virtual
disks. The filesystem mode is entered as soon as a virtual disk is opened.

%\textbf{TODO: DISCUSSION: sollen ganze ordner importiert und exportiert werden
%können? wird dies von der client-seite gehandelt?}

\subsubsection{startup}
The command line client can be started as follows:

\begin{verbatim}
java -jar VFSCore.jar ch.eth.jcd.badgers.vfs.ui.VFSConsole
\end{verbatim}

or by starting \verb|ch.eth.jcd.badgers.vfs.ui.VFSConsole| in eclipse.



This gives a console prompt where the following commands can be used in.

\subsubsection{commands}
Following commands can be used with the command line client in management mode:

\begin{itemize}
  \item{\textbf{create c:\textbackslash path\textbackslash to\textbackslash
  disk.bfs 1024}} creates virtual disk with a maximum quota of 1024 megabytes on the host system. The file may grow up to 1024 megabytes.
  \textbf{TODO}: more parameters are needed (encryption, compression, password if there is encryption)
  \item {\textbf{open c:\textbackslash path\textbackslash to\textbackslash
  disk.bfs}} opens filesystem mode for the given virtual disk
  \item {\textbf{exit}} exits the console program
\end{itemize}

follwing commands can be used in filesystem mode:

\begin{itemize}
  \item {\textbf{ls}} lists the contents of the current directory
  \item {\textbf{pwd}} shows the path to the current directory
  \item {\textbf{df}} shows the usage of the current virtual disk
  \item {\textbf{cd dst}} changes current directory to \textit{dst} which must
  be either a child directory of the current path or ``..''
  \item {\textbf{find searchString}} lists absolute paths of all files
  containing \textit{searchString} in their file name
  \item {\textbf{mkdir dirName}} creates a new directory \textit{dirName} in the
  current path
  \item {\textbf{mkfile fileName}} creates a new empty file \textit{fileName} in
  the current path - this is rather not useful, as the ``import'' creates a
  file with content
  \item {\textbf{rm file}} deletes the entry denoted as \textit{file}, it must
  be a child of the current path
  \item {\textbf{cp src dst}} copies the \textit{src} file to \textit{dst} as a
  child of the current path
  \item {\textbf{mv src dst}} moves the \textit{src} file to \textit{dst}
  \item {\textbf{import ext\_src dst}} imports a \textit{ext\_src} from the
  host system to \textit{dst}
  \item {\textbf{export src ext\_src}} exports a \textit{src} file to the host
  system \textit{ext\_dst}
  \item {\textbf{find searchString}} lists all filesystem entries below the
  current entry containing \textit{searchString}
  \item {\textbf{dispose}} deletes the currently opened virtual disk
  \item {\textbf{close}} closes the filesystem mode, from now on management mode
  commands can be executed
\end{itemize}