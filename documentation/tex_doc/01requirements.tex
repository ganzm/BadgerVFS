\subsection{Requirements}
Below one will find a list of the requirements implemented in this project.

\subsubsection {disk management}
\begin{itemize}
  \item \emph{The virtual disk must be stored in a single file in the working
  directory in the host file system.}
  \item \emph{VFS must support the creation of a new disk with the specified
  maximum size at the specified location in the host file system. }
  \item \emph{VFS must support several virtual disks in the host file system.}
  \item \emph{VFS must support disposing of the virtual disk.}
  \item \emph{VFS must support querying of free/occupied space in the virtual
  disk.}
\end{itemize}  These requirements are met with the classes
\begin{verbatim}
ch.eth.jcd.badgers.vfs.core.VFSDiskManagerImpl
ch.eth.jcd.badgers.vfs.core.config.DiskConfiguration
\end{verbatim} allow creation/deletion and opening of the disk on the host
file system. Clients of the library have to pass a \textit{DiskConfiguration}
to the \textit{VFSDiskManagerImpl} when calling \textit{create} or
\textit{open}
\subsubsection{file management}
\begin{itemize}
  \item \emph{VFS must support creating/deleting/renaming directories and files.}
  \item \emph{VFS must support navigation: listing of files and folders, and
  going to a location expressed by a concrete path.}
  \item \emph{VFS must support moving/copying directories and files, including
  hierarchy.}
  \item \emph{VFS must support importing files and directories from
  the host file system.}
  \item \emph{VFS must support exporting files and directories to the host file
  system.}
\end{itemize} These requirements are met with the classes
\begin{verbatim}
ch.eth.jcd.badgers.vfs.core.VFSEntryImpl
ch.eth.jcd.badgers.vfs.core.VFSPathImpl
ch.eth.jcd.badgers.vfs.core.VFSFileInputStream
ch.eth.jcd.badgers.vfs.core.VFSFileOutputStream
\end{verbatim}The \textit{VFSEntryImpl} allows
copy, move (and rename), delete, listing of children and going to the parent
(navigation). Together with the streams it also supports importing/exporting
to any location clients of the VFS core library wish to. The classes
\begin{verbatim}
ch.eth.jcd.badgers.vfs.ui.VFSConsole
ch.eth.jcd.badgers.vfs.ui.VFSUIController
\end{verbatim} demonstrate this by importing and exporting to the host file
system.

\subsubsection {bonus features}
\begin{itemize}
  \item \emph{Elastic disk: Virtual disk can dynamically grow or shrink,
  depending on its occupied space.}
\end{itemize}
The implementation only allows growing if more files are imported. Shrinking is
not supported.

\begin{itemize}
  \item \emph{Compression, if implemented with 3d party library.}
  \item \emph{Compression, if implemented by hand (you can take a look at
  the arithmetic compression)}
\end{itemize}

The classes
\begin{verbatim}
ch.eth.jcd.badgers.vfs.compression.BadgersLZ77CompressionInputStream
ch.eth.jcd.badgers.vfs.compression.BadgersLZ77CompressionOutputStream
ch.eth.jcd.badgers.vfs.compression.BadgersRLECompressionInputStream
ch.eth.jcd.badgers.vfs.compression.BadgersRLECompressionOutputStream
\end{verbatim}
implement streams that can be wrapped around \textit{VFSFileInputStream} and
\textit{VFSFileOutputStream}. The \textit{DiskConfiguration} allows
to switch compression on and to declare which algorithm shall be chosen. This
allows easy configuration of any 3rd party compression streams (which was not
chosen to implement, because of the implementation of our own compression
algorithm).

\begin{itemize}
  \item \emph{Encryption, if implemented with 3rd party library.}
  \item \emph{Encryption, if implemented by hand.}
\end{itemize}

The classes
\begin{verbatim}
ch.eth.jcd.badgers.vfs.encryption.CaesarInputStream
ch.eth.jcd.badgers.vfs.encryption.CaesarOutputStream
\end{verbatim}
show how encryption can be implemented in the library. It was chosen to
implement encryption similar to compression with streams, which allows easy
configuration via \textit{DiskConfiguration}. These streams are mainly for
demonstration how encryption should work and shall not be used in high
security environments :-)

\begin{itemize}
  \item \emph{Large data: This means, that VFS core can store \& operate amount
  of data, that can't fit to PC RAM ( typically, more than 4Gb).}
\end{itemize}
Having all operations implemented with streams achieves this requirement on the
fly. Manual tests with 6GB files showed pretty efficient import and export to
and from virtual disks.
